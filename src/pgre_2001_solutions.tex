\documentclass[12pt]{article}

\newcommand{\Year}{2001}
\newcommand{\Ident}{GR0177}
\newcommand{\Version}{2.0}

\title{Solutions to \Year Physics GRE}
\author{Jonathan F. Dooley}

\usepackage{lipsum}
\usepackage{pdfpages}
\usepackage{setspace}
\usepackage{lineno} % linenumbers

% no indentation
\setlength{\parindent}{0pt}

\usepackage{graphicx, wrapfig, svg}
\usepackage[caption = false]{subfig}
\usepackage{caption}

\usepackage{color, colortbl}
\definecolor{lgray}{gray}{0.8}

\usepackage{newclude} % remove clearpage, include*

% allow \subsubsection and
\usepackage{titlesec}
\setcounter{secnumdepth}{4}
\setcounter{tocdepth}{4}

% create live links in TOC
\usepackage[hypertexnames=false]{hyperref}
\hypersetup{
  colorlinks = true,
  linkcolor = black,
  anchorcolor = black,
  citecolor = black,
  filecolor = black,
  urlcolor = black,
  pdfnewwindow = true,
  extension = pdf
}

\setlength{\parindent}{0pc}
\setlength{\parskip}{5pt plus1.5pt minus0.5pt} % <- from nmt thesis

\usepackage{lib/jfdgeom}
\usepackage{lib/jfdshorts}
\usepackage{lib/jfdtypeset}
\usepackage{lib/jfdcode}

\usepackage{pgre}
\usepackage{multicol}

\fancyfoot[C]{\thepage}
\fancyfoot[R]{\Ident\xspace Solutions v\Version}

\begin{document}
\TitlePage{\Year}{\Ident}{\Version}

\begin{center}
\boxed{\text{\large Q = Question \hspace{.2in} A = Answer \hspace{.2in}  P+ = Percent Correct in \Year }}
\end{center}
\begin{multicols}{5}
\begin{enumerate}
\item[Q] \begin{tabular}{cc} A & P+\end{tabular}
\item[Q] \begin{tabular}{cc} A & P+\end{tabular}
\item[Q] \begin{tabular}{cc} A & P+\end{tabular}
\item[Q] \begin{tabular}{cc} A & P+\end{tabular}
\item[Q] \begin{tabular}{cc} A & P+\end{tabular}
\end{enumerate}
\end{multicols}

\begin{multicols}{5}
\begin{enumerate}
\item[1] \begin{tabular}{cc} C & 54\%\end{tabular}
\item[2] \begin{tabular}{cc} D & 30\%\end{tabular}
\item[3] \begin{tabular}{cc} D & 71\%\end{tabular}
\item[4] \begin{tabular}{cc} C & 62\%\end{tabular}
\item[5] \begin{tabular}{cc} D & 28\%\end{tabular}
\item[]
\item[6] \begin{tabular}{cc} E & 34\%\end{tabular}
\item[7] \begin{tabular}{cc} B & 89\%\end{tabular}
\item[8] \begin{tabular}{cc} D & 65\%\end{tabular}
\item[9] \begin{tabular}{cc} A & 63\%\end{tabular}
\item[10] \begin{tabular}{cc} A & 53\%\end{tabular}
\item[]
\item[11] \begin{tabular}{cc} A & 28\%\end{tabular}
\item[12] \begin{tabular}{cc} E & 40\%\end{tabular}
\item[13] \begin{tabular}{cc} B & 42\%\end{tabular}
\item[14] \begin{tabular}{cc} C & 27\%\end{tabular}
\item[15] \begin{tabular}{cc} A & 68\%\end{tabular}
\item[]
\item[16] \begin{tabular}{cc} D & 14\%\end{tabular}
\item[17] \begin{tabular}{cc} B & 81\%\end{tabular}
\item[18] \begin{tabular}{cc} A & 45\%\end{tabular}
\item[19] \begin{tabular}{cc} B & 36\%\end{tabular}
\item[20] \begin{tabular}{cc} E & 49\%\end{tabular}

\item[]

\item[21] \begin{tabular}{cc} B & 60\%\end{tabular}
\item[22] \begin{tabular}{cc} A & 54\%\end{tabular}
\item[23] \begin{tabular}{cc} C & 45\%\end{tabular}
\item[24] \begin{tabular}{cc} C & 86\%\end{tabular}
\item[25] \begin{tabular}{cc} E & 49\%\end{tabular}
\item[]
\item[26] \begin{tabular}{cc} C & 30\%\end{tabular}
\item[27] \begin{tabular}{cc} A & 82\%\end{tabular}
\item[28] \begin{tabular}{cc} E & 61\%\end{tabular}
\item[29] \begin{tabular}{cc} C & 63\%\end{tabular}
\item[30] \begin{tabular}{cc} A & 44\%\end{tabular}
\item[]
\item[31] \begin{tabular}{cc} A & 53\%\end{tabular}
\item[32] \begin{tabular}{cc} D & 62\%\end{tabular}
\item[33] \begin{tabular}{cc} D & 31\%\end{tabular}
\item[34] \begin{tabular}{cc} C & 23\%\end{tabular}
\item[35] \begin{tabular}{cc} E & 82\%\end{tabular}
\item[]
\item[36] \begin{tabular}{cc} E & 70\%\end{tabular}
\item[37] \begin{tabular}{cc} D & 36\%\end{tabular}
\item[38] \begin{tabular}{cc} D & 35\%\end{tabular}
\item[39] \begin{tabular}{cc} D & 45\%\end{tabular}
\item[40] \begin{tabular}{cc} D & 40\%\end{tabular}

\item[]

\item[41] \begin{tabular}{cc} E & 66\%\end{tabular}
\item[42] \begin{tabular}{cc} C & 64\%\end{tabular}
\item[43] \begin{tabular}{cc} D & 39\%\end{tabular}
\item[44] \begin{tabular}{cc} D & 54\%\end{tabular}
\item[45] \begin{tabular}{cc} B & 50\%\end{tabular}
\item[]
\item[46] \begin{tabular}{cc} E & 29\%\end{tabular}
\item[47] \begin{tabular}{cc} B & 46\%\end{tabular}
\item[48] \begin{tabular}{cc} C & 57\%\end{tabular}
\item[49] \begin{tabular}{cc} E & 61\%\end{tabular}
\item[50] \begin{tabular}{cc} B & 50\%\end{tabular}
\item[]
\item[51] \begin{tabular}{cc} B & 45\%\end{tabular}
\item[52] \begin{tabular}{cc} C & 12\%\end{tabular}
\item[53] \begin{tabular}{cc} B & 32\%\end{tabular}
\item[54] \begin{tabular}{cc} C & 77\%\end{tabular}
\item[55] \begin{tabular}{cc} E & 62\%\end{tabular}
\item[]
\item[56] \begin{tabular}{cc} D & 54\%\end{tabular}
\item[57] \begin{tabular}{cc} A & 68\%\end{tabular}
\item[58] \begin{tabular}{cc} B & 58\%\end{tabular}
\item[59] \begin{tabular}{cc} B & 87\%\end{tabular}
\item[60] \begin{tabular}{cc} D & 55\%\end{tabular}

\item[]

\item[61] \begin{tabular}{cc} C & 18\%\end{tabular}
\item[62] \begin{tabular}{cc} A & 35\%\end{tabular}
\item[63] \begin{tabular}{cc} D & 52\%\end{tabular}
\item[64] \begin{tabular}{cc} A & 56\%\end{tabular}
\item[65] \begin{tabular}{cc} D & 44\%\end{tabular}
\item[]
\item[66] \begin{tabular}{cc} D & 33\%\end{tabular}
\item[67] \begin{tabular}{cc} E & 19\%\end{tabular}
\item[68] \begin{tabular}{cc} E & 51\%\end{tabular}
\item[69] \begin{tabular}{cc} B & 26\%\end{tabular}
\item[70] \begin{tabular}{cc} B & 53\%\end{tabular}
\item[]
\item[71] \begin{tabular}{cc} D & 32\%\end{tabular}
\item[72] \begin{tabular}{cc} E & 39\%\end{tabular}
\item[73] \begin{tabular}{cc} D & 43\%\end{tabular}
\item[74] \begin{tabular}{cc} D & 50\%\end{tabular}
\item[75] \begin{tabular}{cc} E & 57\%\end{tabular}
\item[]
\item[76] \begin{tabular}{cc} C & 49\%\end{tabular}
\item[77] \begin{tabular}{cc} E & 44\%\end{tabular}
\item[78] \begin{tabular}{cc} E & 52\%\end{tabular}
\item[79] \begin{tabular}{cc} D & 69\%\end{tabular}
\item[80] \begin{tabular}{cc} D & 28\%\end{tabular}

\item[]

\item[81] \begin{tabular}{cc} B & 50\%\end{tabular}
\item[82] \begin{tabular}{cc} D & 16\%\end{tabular}
\item[83] \begin{tabular}{cc} C & 30\%\end{tabular}
\item[84] \begin{tabular}{cc} D & 26\%\end{tabular}
\item[85] \begin{tabular}{cc} A & 25\%\end{tabular}
\item[]
\item[86] \begin{tabular}{cc} E & 24\%\end{tabular}
\item[87] \begin{tabular}{cc} A & 42\%\end{tabular}
\item[88] \begin{tabular}{cc} C & 42\%\end{tabular}
\item[89] \begin{tabular}{cc} E & 37\%\end{tabular}
\item[90] \begin{tabular}{cc} A & 33\%\end{tabular}
\item[]
\item[91] \begin{tabular}{cc} B & 41\%\end{tabular}
\item[92] \begin{tabular}{cc} E & 45\%\end{tabular}
\item[93] \begin{tabular}{cc} C & 42\%\end{tabular}
\item[94] \begin{tabular}{cc} E & 29\%\end{tabular}
\item[95] \begin{tabular}{cc} A & 42\%\end{tabular}
\item[]
\item[96] \begin{tabular}{cc} E & 13\%\end{tabular}
\item[97] \begin{tabular}{cc} E & 20\%\end{tabular}
\item[98] \begin{tabular}{cc} A & 72\%\end{tabular}
\item[99] \begin{tabular}{cc} D & 20\%\end{tabular}
\item[100] \begin{tabular}{cc} B & 72\%\end{tabular}

\item[]
\end{enumerate}
\end{multicols}
\clearpage

\Problem{1}{C}{%
The acceleration on the pendulum is the sum of the tangential and centripetal accelerations: $\vec{a} = \vec{a}_{tan} + \vec{a}_{cent}$ where

\begin{gather}
\vec{a}_{tan} = \frac{d\vec{v}}{dt} = \frac{d (\vec{\omega} r)}{d t} = r \frac{d\vec{w}}{dt} = r \vec{\alpha}\\
\vec{a}_{cent} = \frac{\vec{v}^{2}}{r} = \frac{\vec{\omega}^{2} r^{2}}{r} = \vec{\omega}^{2} r
\end{gather}
\\
At point $c$, $\vec{\omega} = const$ so the angular acceleration $\vec{\alpha} = \frac{d \vec{\omega}}{dt} = 0$. Therefore, $\vec{a} = \vec{a}_{cent}$. $\vec{a}_{cent}$ points towards the axis of rotation (pivot point), which eliminates (A), (B), and (D).\\
\\
At points $a$ and $e$, $\vec{\omega} = 0$ so that $\vec{a} = \vec{a}_{tan}$.
}

\Problem{2}{D}{%
This is a force balancing problem:

\begin{gather}
F_{cent} = F_{fr}\nonumber\\
m \hspace{.01in} \omega^{2} r = m g \mu_{s} \hspace{.1in} \rightarrow \hspace{.1in} r =\frac{g \mu_{s}}{\omega^2} \nonumber
\end{gather}
\\
for this equation to work we must correct the dimensions of $\omega$:

\begin{gather}
\omega= (33.3\Units{rev/min}) \left(   \frac{2 \pi\Units{rads}}{1\Units{rev}} \right)   \left(\frac{1\Units{min}}{60\Units{s}} \right) = 3.48\Units{rads/s}\nonumber\\
\nonumber\\
\therefore \hspace{.1in}r = \frac{10\Units{m s$^{-2}$} \cdot 0.30}{(3.48\Units{rad/s})^{2}} = 0.24\Units{m}\nonumber
\end{gather}
}


\Problem{3}{D}{%
Recall Kepler's Third Law:

\begin{gather}
T^{2} = \frac{4 \pi^{2} a^{3}}{G M}
\end{gather}
\\
where $T$ is the period, $a$ is the semi-major axis, M is the mass of the planet, and G is Newton's gravitational constant. For a circular orbit, both the semi-major and semi-minor axes are equal to $R$.

\begin{gather}
\therefore \hspace{.1in} T^{2} \propto R^{3} \hspace{.1in} \rightarrow \hspace{.1in} \boxed{T \propto R^{3/2}}\nonumber
\end{gather}
}

\Problem{4}{C}{%
From conservation of momentum, we can calculate the final velocity of the two particle system, $v_{f}$:

\begin{align}
p_{0} &= p_{f} \nonumber\\
2m v_{0} + 0 &= (2m + m) v_{f}\nonumber\\
\rightarrow \hspace{.1in} v_{f} &= \frac{2}{3} v_{0} \nonumber
\end{align}
\\
The fraction of initial kinetic energy lost in the collision can be found by dividing kinetic energy lost, $ T_{0} - T_{f} $, by the initial kinetic energy, $T_{0}$.

\begin{align}
\frac{T_{0} - T_{f}}{T_{0}} &=  \frac{   \frac{1}{2} (2m) v_{0}^{2} - \frac{1}{2} (3m) v_{f}^{2}   }{\frac{1}{2} (2m) v_{0}^{2}}\nonumber\\
&=\frac{ (2m) v_{0}^{2} - (3m) \left(  \frac{2}{3} v_{0} \right)^{2}   }{(2m) v_{0}^{2}}\nonumber\\
&= \frac{2 - \frac{12}{9} }{2} = \frac{\frac{18}{9} - \frac{12}{9}}{\frac{18}{9}} = \frac{6}{18} = \boxed{\frac{1}{3}}\nonumber
\end{align}
}

\Problem{5}{C}{%
The Equipartition Theorem states that

\begin{gather}
\langle  E  \rangle = \frac{f}{2} k_{B}T
\end{gather}
\\
Where $k_{B}$ is Boltzmann's constant, $T$ is the temperature, and $f$ is the degrees of freedom. The degrees of freedom for an oscillator in 1D, 2D, and 3D is found using the equation $f =$ (translational degrees) $+$ (rotational degrees) $+$ (vibrational degrees):\\

\hspace{2.7in}1D : $f = 2 + 0 + 0 = 2$

\hspace{2.7in}2D : $f = 2 + 1 + 1 = 4$

\hspace{2.7in}3D : $f = 3 + 2 + 1 = 6$\\
\\
Therefore, our three dimensional oscillator has average energy

\begin{gather}
\langle  E  \rangle = \frac{6}{2} k_{B}T = \boxed{ 3 k_{B} T  }\nonumber
\end{gather}
}

\Problem{6}{E}{%
Generally, in a closed cycle, an adiabatic process does less work than an isothermal process. To see this, note that on a PV diagram an isothermal expansion follows the curve $P \propto V^{-1}$ while an adiabatic expansion follows the curve $P \propto V^{-\gamma}$. Since $\gamma$ of an ideal monoatomic gas is greater then 1 ($\gamma = \frac{5}{3}$), the area under the isothermal curve (read: $W_{i}$) is greater then the area under the adiabatic curve ($W_{a}$). And both are greater than zero.

\begin{align}
\therefore \hspace{.1in} \boxed{0 < W_{a} < W_{i}}\nonumber
\end{align}
}

\Problem{7}{B}{%
Magnets with the same polarity repel one another. Therefore we can eliminate (A), (D), and (C).
\\\\
At small distances, the magnetic field lines are normal to the surface of the magnet and therefore we can eliminate (E) (Notice that (E) is the magnetic field produced by a current carrying wire, not two magnets).
}

\Problem{8}{D}{%
"A charge located above a grounded conduction plate" is almost always a method of images problem. Therefore we must picture an equal and opposite charge located a distance $L$ \textbf{below} the plate.
\\\\
The imagined charge is $-Q$. Therefore a net charge of $-Q$ is induced on the plate.
}


\Problem{9}{A}{%
Imagine a circle with an infinite number of charges of magnitude $q$ placed along its circumference. This is a model of a 2D hollow conductor which has an internal electric field equal to zero. Now remove all but 5 equally spaced charges. The symmetry of the system has the electric field vectors cancelling out with one another.
\\\\
Do not get tricked into adding up all five charges and choosing answer (C). This solution does not consider the symmetry and, therefore, the destructive superposition of the charges.
}


\Problem{10}{A}{%
The equivalent capacitance of capacitors in series and in parallel are

\begin{align}
\frac{1}{C_{eq}} = \frac{1}{C_{1}} + \frac{1}{C_{2}} + ... \hspace{.1in} \text{(Series)}\\
C_{eq} = C_{1} + C_{2} + ... \hspace{.1in} \text{(Parallel)}
\end{align}
\\
Therefore the equivalent capacitance of our capacitors is

\begin{align}
C_{eq} =\left(  \frac{1}{3\Units{\micro F}} + \frac{1}{6\Units{\micro F}}  \right)^{-1} = \frac{6\Units{\micro F}}{3\Units{\micro F}} = 2{\micro F} \nonumber
\end{align}
\\
The energy stored in a capacitor, $U$, is

\begin{gather}
U = \frac{1}{2} C_{eq} V^{2} =
\frac{1}{2} (2\e{-6}\Units{F}) (300\Units{V})^{2} =
\boxed{0.09\Units{J}}\nonumber
\end{gather}
}


\Problem{11}{A}{%
We must first find the location of the image produced by the 1st lens ($f_{1} = 20\Units{cm}$):

\begin{gather}
\label{eq:lens_pos} \frac{1}{f_{1}} = \frac{1}{s_{1}} + \frac{1}{s_{1}'}
\end{gather}

\begin{align}
\rightarrow \hspace{.1in} s_{1}' &= \left(   \frac{1}{20\Units{cm}} - \frac{1}{40\Units{cm}}  \right)^{-1} \nonumber\\
&= \left(   \frac{2}{40\Units{cm}} - \frac{1}{40\Units{cm}}  \right)^{-1} = 40\Units{cm}\nonumber
\end{align}
\\
The fact that $s_{1}'$ is a positive value tells us that the image is on the opposite side of the lens from the object (right side of lens). \\
\\
Now we look at the image produced by the 2nd lens. The object is no longer at $O$ but instead it is $s_{2} = 10\Units{cm}$ to the right of the 2nd lens (since it is $40\Units{cm}$ to the right of the 1st lens). By convention, this means that the equation for the image produced is

\begin{align}
\label{eq:lens_neg} \frac{1}{f_{2}} &= \frac{1}{s_{2}'} - \frac{1}{s_{2}}
\end{align}
\\
(Note the difference between equations \ref{eq:lens_pos} and  \ref{eq:lens_neg}) This is because the original object is to the right both lenses, so an object on the left would have a negative distance. We can now find the location of $s_{2}'$:

\begin{align}
\rightarrow \hspace{.1in} s_{2}' &= \left(   \frac{1}{f_{2}} + \frac{1}{s_{2}}   \right)^{-1}\nonumber\\
&= \left(   \frac{1}{10\Units{cm}} + \frac{1}{10\Units{cm}}   \right)^{-1} = \left(   \frac{2}{10\Units{cm}} \right)^{-1}  = 5\Units{cm}\nonumber
\end{align}
\\
Once again, because the produced image is positive, the location of the image is to the right of the lens.
\\\\
}


\Problem{12}{E}{%
The mirror equations states that

\begin{align}
\frac{1}{f} &= \frac{1}{s} + \frac{1}{s'}\\
\nonumber\\
\rightarrow  \hspace{.1in} \frac{1}{s'} &= \frac{1}{f}  - \frac{1}{s} \nonumber
\end{align}
\\
(same as \ref{eq:lens_pos}). Because $s < f$ we can conclude that $1/s > 1/f$. Therefore,

\begin{align}
\frac{1}{s'} &= \frac{1}{f}  - \frac{1}{s}  < 0\nonumber
\end{align}
\\
Which means that the image is not on the reflecting side of the mirror and must be at point $V$. This logic comes from the fact that, by convention, $s'$ is positive if produced on the same side as the object.
\\\\
}

\Problem{13}{B}{%
The angular separation of two point sources, $\theta$, is related to the the wavelength, $\lambda$, and telescope diameter, $D$ by the Rayleigh criterion:

\begin{gather}
D \sin{(\theta)} = 1.22 \lambda
\end{gather}
\\
Because $\theta$ is small be can approximate $\sin{(\theta)} \approx \theta$ (only true for \Units{rads}).\\
\\
We are asked to find the diameter needed for a telescope to resolve two stars with $\theta = 3\e{-5}\Units{rad}$  and $\lambda = 600\Units{nm}$. Plugging these values into the Rayleigh criterion and solving for $D$ we get

\begin{align}
D &= 1.22 \cdot \frac{ 600\Units{nm}}{ 3\e{-5}\Units{rad}}  = 1.22\cdot  \frac{6\e{-7}\Units{m}}{3\e{-5}\Units{rad}}\nonumber\\
\nonumber\\
&= 1.22\cdot 2\e{-2}\Units{m} = 2.44\e{-2}\Units{m} = 0.024\Units{m} \sim \boxed{2.5\Units{cm}}\nonumber
\end{align}
}


\Problem{14}{C}{%
This is a simple application of the point source illumination formula where $\sigma$ is the cross sectional area of the detector and $A$ is the area of the imagined sphere with $R = 1\Units{m}$

\begin{align}
I &= \frac{\sigma}{A} \\
\nonumber\\
&= \frac{\pi R^{2}} {4 \pi l^{2}} = \frac{R^{2}} {4 l^{2}} = \frac{(4\Units{cm})^{2}} {4 \cdot (100\Units{cm})^{2}} \nonumber\\
\nonumber\\
&= \frac{16}{4000} = \frac{4}{1000} = \boxed{4\e{-4}}\nonumber
\end{align}
}


\Problem{15}{A}{%
The most precise measurement could be incorrect, as long as the results of many measurements is very close the the average measurement. Therefore, the class that made the most precise measurement is the one with the smallest range of heights.
}

\Problem{16}{D}{%
A Poisson distribution describes radioactive phenomenon. In a poisson process the standard deviation, $\sigma$, can be found by taking the square root of the average, $\overline{x}$:

\begin{gather}
\overline{x} = \frac{3+0+2+1+2+4+0+1+2+5}{10} = \frac{20}{10} = 2\Units{counts/s}\nonumber\\
\nonumber\\
\therefore \hspace{.1in} \sigma = \sqrt{2} \nonumber
\end{gather}
\\
Now, the error on the mean is generally taken be to the standard deviation divided by the square root of the number of measurements, $N$:

\begin{gather}
\text{uncert} \hspace{.05in} = \frac {\sigma}{\sqrt{N}}
\end{gather}
\\
We need to find $N$ when the uncertainty is 1\% of the mean. \textbf{Be careful}  - do not set the uncertainty equal to $0.01$! Because the uncertainty is 1\% of the mean we need to use $0.01 \cdot \overline{x} = 0.02$:

\begin{gather}
0.02 = \frac {\sqrt{2}}{\sqrt{N}} \hspace{.1in} \rightarrow \hspace{.1in} 4\e{-4} = \frac{2}{N} \hspace{.1in} \rightarrow \hspace{.1in} N = \frac{2}{4\e{-4}} = \frac{1}{2\e{-4}} = 0.5\e{4} = \boxed{5000\Units{s}} \nonumber
\end{gather}
}


\Problem{17}{B}{%
The ground state means that each of the energy levels in the first few shells is filled by an electron (to the given maximum of 15 electrons). The electron configuration can be determined using the following diagram:

\begin{figure}[ht]
\includegraphics[width = .50\textwidth]{images/electron_config.png}\centering
\end{figure}
 \noindent The $s$ orbital can hold 2 electrons, $p$ can hold 6, $d$ can hold 10, and $f$ can hold 14. Therefore, the ground state electron configuration for phosphorus is

 \begin{gather}
 \boxed{1s^{2} 2s^{2} 2p^{6} 3s^{2} 3d^{3}}\nonumber
 \end{gather}
 \\
 Notice that, for larger electron numbers, $4s$ comes before $3d$.
}



\Problem{18}{A}{%
The key to this problem is to picture this system as a hydrogen-like atom (one electron) with two protons ($Z = 2$). The ionization energy of a hydrogen-like atom can be found using the equation

\begin{gather}
E_{n} = \frac{Z^{2}}{n^{2}} 13.6\Units{eV}
\end{gather}
\\
where $n$ is the state of the system.

\begin{gather}
\therefore \hspace{.1in} E_{1} = Z^{2} \cdot 13.6\Units{eV} = 54.4\Units{eV}\nonumber
\end{gather}
\\
The problem tells us that the energy to remove both electrons from the He atom is $79.0\Units{eV}$. Therefore the energy required to remove one electron is

\begin{gather}
E_{ion} =  79.0\Units{eV} - 54.4\Units{eV} = \boxed{24.6\Units{eV}}\nonumber
\end{gather}
}


\Problem{19}{B}{%
The primary source of the Sun's energy is the Proton-Proton Chain (PP Chain). in this reaction, \textbf{four H\textsuperscript{1}} atoms are fused an eventually create \textbf{one He\textsuperscript{4}} (I say "eventually" because there are 3 reactions in the PP Chain). From the mass-energy equivalence equation and conservation of energy, we know that the energy released during the PP Chain must be the difference in mass between four hydrogen atoms and one helium atom.
}

\Problem{20}{E}{%
This is pure fact recall. Bremsstrahlung refers to the electromagnetic radiation produced by the acceleration or deceleration of a charged particle (usually electrons) after passing through the electric and magnetic fields of a nucleus. This is a smooth, continuous X-ray spectra.
}

\Problem{21}{B}{%
The Rydberg formula for hydrogen is

\begin{gather}
f = \frac{c}{\lambda} = R \left(  \frac{1}{n_{f}^{2}} - \frac{1}{n_{i}^{2}}  \right)
\end{gather}
\\
Where $R$ is The Rydberg Constant and completely unnecessary for the purposes of this problem. Let us say that $\lambda '$ is the wavelength for Lyman-$\alpha$ and $\lambda$ is the wavelength for Balmer-$\alpha$

\begin{align}
\frac{\lambda' }{\lambda} &=  \frac{\left(  \frac{1}{n_{f}^{2}} - \frac{1}{n_{i}^{2}}  \right)}{\left(  \frac{1}{n_{f}'^{2}} - \frac{1}{n_{i}'^{2}}  \right)} = \frac{\left(   \frac{1}{4} - \frac{1}{9}   \right)   }{\left(    \frac{1}{1} -\frac{1}{4}   \right)} = \frac{\left(   \frac{9}{36} - \frac{4}{36}   \right)   }{\left(   \frac{3}{4}   \right)} = \frac{\frac{5}{36}}{\frac{3}{4}} = \frac{20}{108} = \boxed{\frac{5}{27}} \nonumber
\end{align}
\\
Make sure that you dont invert your fraction! Solution (E) is incorrect because it is Balmer-$\alpha$ to Lyman-$\alpha$ radiation and an excellent example of a common GRE a trap answer.
}


\Problem{22}{A}{%
The angular momentum of the moon is the same at every point in it's orbit. The equation for angular momentum, $L$, is

\begin{gather}
L = r \times p = m (r \times v) = m r v \sin{(\theta)}
\end{gather}
\\
The minimum, $r_{p}$, and maximum, $r_{a}$, distances (perigee and apogee) of the orbit correspond to a tangential velocity for which $\theta \hspace{.02in} = \hspace{.03in} 90^{^\circ}$.

\begin{gather}
L = m \cdot v_{a} \cdot r_{a} = m \cdot v_{p} \cdot r_{p}\
\end{gather}
\\
Obviously, we can cancel out the moon's mass, $m$, in the above conservation equation. Therefore we are unable to solve for this mass.\\
\\
Interesting tidbit: when the Soviet Union launched sputnik in 1957, American scientists were unable to determine the satellite's mass for this very reason.
\\\\
}


\Problem{23}{C}{%
The particle is moving with a tangential velocity $v_{\perp} = 10\Units{m/s}$ and the acceleration is $a_{rad} = 10\Units{m/s$^{2}$}$. Since these values have the same magnitude, the angle between the velocity and acceleration vectors is $45^{\circ}$ (picture this as a 45-45-90 triangle).
}

\Problem{24}{C}{%
Gravity only applies in the vertical, $\hat{y}$, direction. Therefore $v_{x}$ is constant with respect to time. This eliminates (A) and (E).\\
\\
Since gravity is working in the $-\hat{y}$ direction while the stone is moving (partly) in the $+\hat{y}$ direction: $v_{y}$ starts positive and decreases with time. At the top of the stone's projectile motion $v_{y} = 0$. Then the stone begins to fall back to the ground so that $v_{y}$ is negative.
}


\Problem{25}{E}{%
The parallel axis theorem is an invaluable tool for computing the moment of inertia of systems built out of smaller pieces whose moments of inertia are known. The moment of inertia about any axis parallel to the center of mass axis is given by the equation

\begin{gather}
\label{eq:par_ax}I = I_{CM} +m r^{2}
\end{gather}
\\
One of these pennies (the middle one) has the same moment of inertia as a disk with radius $r$ and mass $m$. The other 6 require the the parallel axis theorem since the axis of rotation is $2r$ away from each penny's center of mass. The moment of inertia for a disk is $I = \frac{1}{2} m r^{2}$.

\begin{align}
I &= \frac{1}{2} m r^{2} + 6 \left(  \frac{1}{2}mr^2 + m (2r)^{2}  \right) = \frac{1}{2}mr^{2} + 6m r^{2} \left(  \frac{1}{2}  + \frac{8}{2}  \right)  = \boxed{\frac{55}{2} m r^{2}} \nonumber
\end{align}
}

\Problem{26}{C}{%
The moment of inertia for a rod is

\begin{gather}
I = \frac{1}{12} M L^{2}
\end{gather}
\\
We must use the parallel axis theorem (equation \ref{eq:par_ax}) again to to find the moment of inertia about the pivot:

\begin{gather}
I = \frac{1}{12} M L^{2} + M \left(   \frac{L}{2}  \right)^{2} = ML^{2} \left(  \frac{1}{12}  + \frac{3}{12}  \right) = \frac{1}{3}M L^{2}\nonumber
\end{gather}
\\
Now we need to find the speed of the pendulum once the free end hits the ground. The best and easiest way to do this is through use of conservation laws. The free end is $\frac{L}{2}$ above the rod's center of mass and $\omega = \frac{v}{L}$

\begin{align}
m g h &= \frac{1}{2} I \omega^{2}  \nonumber\\
M g \frac{L}{2} &= \frac{1}{2} \left(  \frac{1}{3} M L^{2}  \right) \frac{v^{2}}{L^{2}}   \nonumber\\
gL &= \frac{1}{3}v^{2} \hspace{.1in} \rightarrow \hspace{.1in} \boxed{v = \sqrt{3gL}}  \nonumber
\end{align}
}


\Problem{27}{A}{%
The eigenvalues of a Hermitian operator are always real.\\
\\
Proof: The eigenvalue equation is

\begin{align}
A \ket{\psi} &= \lambda \ket{\psi}\nonumber\\
\therefore \hspace{.1in} A^{\dag} \ket{\psi} &= \lambda^{*} \ket{\psi}\nonumber
\end{align}
\\
Subtracting the second equation from the first we get

\begin{gather}
( A - A^{\dag} )\ket{\psi} = ( \lambda - \lambda^{*} ) \ket{\psi}\nonumber
\end{gather}
The left hand side of the above equation is equal to zero since $A = A^{\dag}$. Therefore, $\lambda = \lambda^{*}$ which means that $\lambda$ is real-valued.
\\\\
}


\Problem{28}{E}{%
Lets say that states n and m are orthonormal: $\braket{ n  | m } = 0$ and $\braket{ n  | n } = \braket{ m  | m } = 1$. Therefore:

\begin{align}
\braket{\psi_{1} | \psi_{2}} &= 0\nonumber\\
&= 5 \braket{ 1 | 1 }  + 15 \braket{ 2 | 2 }  + 2x \braket{ 3 | 3 } \nonumber\\
0 &= 20 + 2x \hspace{.1in} \rightarrow \hspace{.1in} \boxed{x = - 10}\nonumber
\end{align}
\\\\
}

\Problem{29}{C}{%
Recall the Dirac notation identity:

\begin{gather}
\bra{\psi} \hat{A} \ket{\psi} =  \lambda \braket {\psi  | \psi} = \lambda
\end{gather}
\\
Where $\lambda$ is an eigenvalue of $\hat{A}$. From this we can see that, using our given eigenvalues, we get

\begin{align}
\bra{\psi_{-1}} \hat{0} \ket{\psi_{-1}} &= -1 \braket {\psi_{-1}  | \psi_{-1}} = -1\nonumber\\
\bra{\psi_{1}} \hat{0} \ket{\psi_{1}} &= 1 \braket {\psi_{1}  | \psi_{1}} = 1\nonumber\\
\bra{\psi_{2}} \hat{0} \ket{\psi_{2}} &= 2 \braket {\psi_{2}  | \psi_{2}} = 2\nonumber
\end{align}
\\
Therefore,

\begin{gather}
\bra{\psi} \hat{O} \ket{\psi} = \frac{1}{6} \braket {\psi_{-1}  | \psi_{-1}} + \frac{1}{2} \braket {\psi_{1}  | \psi_{1}} + \frac{1}{3} \braket {\psi_{2}  | \psi_{2}} = -\frac{1}{6} + \frac{1}{2} + \frac{2}{3} = \boxed{1}\nonumber
\end{gather}
}


\Problem{30}{A}{%
This kind of problem begs for the process of elimination. Radial wave functions must be normalizable and must go to zero at infinity. Therefore:
\begin{description}
\item[I.] Works. it is normalizable (square integrable) and goes to zero at infinity.
\item[II.] Doesn't work. Sinusoidal functions do not converge and therefore do not go to zero at infinity.
\item[III.] Doesn't work. This function is not normalizable.
\end{description}
}


\Problem{31}{A}{%
The ground state energy of hydrogen is

\begin{gather}
E_{0,H} = 13.6\Units{eV} \propto \mu
\end{gather}
\\
where $\mu$ is the reduced mass of the system given by the equation

\begin{gather}
\mu = \frac{m_{1} \cdot m_{2}}{m_{1} + m_{2}}
\end{gather}
\\
For a normal hydrogen atom, $\mu \approx m_{e}$ (because $m_{p} \gg m_{e}$). For positronium:

\begin{gather}
\mu = \frac{m_{e} \cdot m_{e}}{m_{e} + m_{e}} = \frac{me}{2}\nonumber
\end{gather}
\\
or 1/2 the reduced mass of hydrogen. Therefore the ground state energy of positronium is

\begin{gather}
E_{0,p} = \frac{13.6\Units{eV}}{2} = 6.8\Units{eV}\nonumber
\end{gather}
\\
Now, we are looking for the energy of the photon emitted when the positronium makes the transition from $n = 3$ to $n = 1$:

\begin{gather}
E_{\gamma} = 6.8\Units{eV} \cdot \left(  \frac{1}{n^{2}_{f}} - \frac{1}{n^{2}_{i}} \right) =
6.8\Units{eV}  \cdot \left(  \frac{9}{9} - \frac{1}{9} \right) =
6.8\Units{eV}  \cdot \left(  \frac{8}{9}   \right) =
\frac{54\Units{eV} }{9} =
\boxed{6\Units{eV}} \nonumber
\end{gather}
}


\Problem{32}{D}{%
 The relativistic energy-momentum relation is

\begin{gather}
\label{eq:rel-energy} E^{2} = (pc)^{2} + (m_{0} c^{2})^{2}
\end{gather}
\\
The given particle has $m_{0} = m$ and $E = 2mc^{2}$. Solving for $p$ in the above equation

\begin{gather}
p^{2} = \frac{E^{2}}{c^{2}} -  m_{0}^{2}c^{2}  =  \frac{(2mc^{2})^{2}}{c^{2}} -  m^{2}c^{2}  = \frac{4m^{2}c^{4}}{c^{2}} -  m^{2}c^{2} = 3m^{2}c^{2}\nonumber\\
\nonumber\\
\therefore \hspace{.1in} p = \sqrt{3m^{2}c^{2}} = \boxed{\sqrt{3} m c}\nonumber
\end{gather}
}

\Problem{33}{D}{%
Let us say that the particle's rest frame is $S'$ and the lab frame is $S$. The Lorentz Invariance Formula is:

\begin{gather}
\left(  \Delta x  \right)^{2}  - \left(  c \Delta t  \right)^{2}  = \left(  \Delta x'  \right)^{2}  - \left(  c \Delta t'  \right)^{2}
\end{gather}
\\
We are given that $\Delta t' = 10^{-8}\Units{s}$, $\Delta x = 30\Units{m}$, and $\Delta x' = 0\Units{m}$ (since $S'$ is the particle's rest frame). So let us solve for $\Delta t$:

\begin{gather}
\Delta t = \sqrt{\frac{\Delta x^{2}}{c^{2}} + \Delta t'^{2}}\nonumber
\end{gather}
\\
we are solving for the speed of the pion which can be found using the equation

\begin{align}
v &= \frac{\Delta x}{\Delta t}\\
&= \Delta x \left(   \frac{\Delta x^{2}}{c^{2}} + \Delta t'^{2}   \right)^{-\frac{1}{2}} = c \Delta x \left(   \Delta x^{2} + c^{2}\Delta t'^{2}   \right)^{-\frac{1}{2}}\nonumber\\
\nonumber\\
&= \frac{c \cdot 30\Units{m}}{\sqrt{900\Units{m$^{2}$} + (3\e{8}\Units{m/s})^{2} \cdot (10^{-16}\Units{s$^{2}$})}} =
c \cdot\sqrt{    \frac{900\Units{m$^{2}$}}{900\Units{m$^{2}$} + 9\Units{m$^{2}$} }     }\nonumber\\
\nonumber\\
 &= c \cdot \sqrt{\frac{100}{101}} \approx \boxed{2.98\Units{m/s}}\nonumber
\end{align}
}


\Problem{34}{B}{%
In special relativity there are three different classifications of invariant intervals:

\begin{description}
\item[I.] \textbf{Space-like}: $\Delta s^{2} > 0$.  These events can occur simultaneously for some observers but are separated in space.

\item[II.] \textbf{Time-like}: $\Delta s^{2} < 0$. These events can occur at the same location but are separated by time.

\item[III.] \textbf{Light-like}: $\Delta s^{2} = 0$. Only something traveling at the speed of light could be present at both events.
\end{description}
In this problem we are being asked about when the interval is \textbf{Space-like}.

\begin{align}
 0 < \Delta s^{2}   =  \Delta x^{2} - c^{2} \Delta t^{2} \hspace{.1in} \rightarrow \hspace{.1in}
 \Delta x^{2} <  c^{2} \Delta t^{2} \hspace{.1in} \rightarrow \hspace{.1in}
 \boxed{\left|   \frac{\Delta x}{\Delta t} \right| < c}\nonumber
\end{align}
}

\Problem{35}{E}{%
\section{\textsc{Problem 35:}} The Stefan-Boltzmann Law is

\begin{gather}
dP = \epsilon \sigma T^{4} dA \hspace{.1in} \rightarrow \hspace{.1in} \frac{dP}{dA} \propto T^{4}
\end{gather}
\\
Therefore, if the temperature increases by a factor of 3, the energy radiated per second per unit area is

\begin{gather}
\frac{dP}{dA} \propto \boxed{3^{4} = 81}\nonumber
\end{gather}
}


\Problem{36}{E}{%
Process of elimination:
\begin{description}
\item[(A)] True. This is the definition of an adiabatic process.
\item[(B)] True. There is no change in entropy in a quasi-static expansion.
\item[(C)] True.  $\Delta U = \Delta Q - \Delta W$ while $\Delta Q=0$ for an adiabatic process and $\Delta W = \int P dV$. Therefore, $\Delta U =  -\int P dV$
\item[(D)] True. See above.
\item[(E)] False. $\Delta W = P \Delta V = N k  \Delta T$. Do not confuse adiabatic ($\Delta Q = 0$) and isothermal  ($\Delta T = 0$) processes. \\
\end{description}
}


\Problem{37}{D}{%
The area inside a closed loop on a PV diagram is the work done by the process. The direction of the cycle determines the sign of the work (convention: clockwise is positive, counterclockwise is negative). From this information we can eliminate (A), (B), and (C).\\
\\
we can determine the volume at $B$, $V_{B}$, using the ideal gas law $P V = n R T$. Because $BC$ is an isothermal (constant temperature) process

\begin{gather}
P_{B} V_{B} = const =
P_{C} V_{C} \hspace{.1in} \rightarrow \hspace{.1in} V_{B} =
\frac{P_{C} V_{C}}{P_{B}} =
\frac{500\Units{kPa} \cdot 2\Units{m$^{3}$}}{200\Units{kPa}} =
5\Units{m$^{3}$}\nonumber
\end{gather}
\\
Now, let us approximate the cycle as a right triangle with height $P = 500\Units{kPa}-200\Units{kPa} = 500\Units{kPa}$ and base $V = 5\Units{m$^{3}$} - 2\Units{m$^{3}$} = 3\Units{m$^{3}$}$. The area of this triangle (read: magnitude of work done) would be

\begin{gather}
A = |W| = \frac{1}{2} (3\Units{m$^{3}$}) (500\Units{kPa}) = 450\Units{kJ}
\end{gather}
\\
Because the area of the actual process is less than the area of our approximated triangle we can eliminate (E).
}


\Problem{38}{D}{%
The current is maximized when the resonant frequency is reached. The resonant frequency is reached when the complex impedance is 0:

\begin{gather}
X_{C} - X_{L} = \frac{1}{\omega C} - \omega L = 0 \hspace{.1in} \rightarrow \hspace{.1in} \frac{1}{\omega C} = \omega L \hspace{.1in} \rightarrow \hspace{.1in} \omega = \frac{1}{\sqrt{LC}}
\end{gather}
\\
Solving for $C$ and plugging in the given values for $\omega$ and $L$ we get

\begin{gather}
C = \frac{1}{L \omega^{2}} =  \frac{1}{25\Units{mH} (1000\Units{rads/s})^{2}}  = 4\e{-5}\Units{F} = \boxed{40\Units{\micro F}} \nonumber
\end{gather}
}

\Problem{39}{D}{%
A high-pass filter is a series combination of a \textbf{Capacitor followed by a Resistor} or a \textbf{Resistor followed by an Inductor} (CR or RL)\\
\\
The inverse is a low-pass filter: a  series combination of a \textbf{Resistor followed by a Capacitor} or an \textbf{Inductor followed by a Resistor} (RC or LR)
}

\Problem{40}{D}{%
We can eliminate (B), (C) and (E) based on the fact that the time constant of the ciruit is

\begin{gather}
\tau = \frac{L}{R} = \frac{10\e{-3}\Units{H}}{2\Units{$\Omega$}} = 5\e{-3}\Units{s} = 5\Units{ms}
\end{gather}
\\
The voltage with with respect to time can be calculated using the equation

\begin{gather}
V(t) = V_{0} e^{-t/\tau}
\end{gather}
\\
So that at $t=0$ the voltage is $10\Units{V}$
}


\Problem{41}{E}{%
Gauss's law for magnetism $\nabla \cdot \vec{B} = 0$ is the mathematic statement "Magnetic monopoles do not exist." Therefore, we can eliminate (A), (C), and (D) since they do not include II.\\
\\
If magnetic monopoles were to exist, Maxwell's Equations would be symmetric with their counterparts (divergence and curl). Therefore, we would have to change both II (as discussed above) and IV.
}

\Problem{42}{C}{%
The induced current in loops $A$ and $B$ follows the equation

\begin{gather}
\mathcal{E} = -\frac{d \Phi_{B}}{d t} = -A \frac{d B}{dt}
\end{gather}
\\
The minus sign in this equation is a statement that the current inducted opposes the changing field. Therefore as the current loop moves towards $A$ the magnetic field increases at $A$ and the induced current is clockwise. at the same time, the magnetic field decreases at $B$ and the induced current is counterclockwise.
}



\Problem{43}{D}{%
\textbf{Memorize this identity}:

\begin{gather}
[AB, C] = B [A, C] + [B, C] A
\end{gather}
\\
Using this, the given commutator comes out to

\begin{align}
[L_{x}L_{y}, L_{z}] &= L_{y} [L_{x}, L_{z}] + [L_{y}, L_{z}] L_{x}\nonumber\\
&= L_{y} (-i \hbar L_{y}) + (i \hbar L_{x})L_{x} = i \hbar ( L_{x}^{2} -  L_{y}^{2} )\nonumber
\end{align}
}

\Problem{44}{D}{%
Everything that you need is given to you in the problem statement. If

\begin{gather}
E_n = \frac{n^{2} \pi^{2} \hbar^{2}}{2mL^{2}}
\end{gather}
\
then
\begin{align}
E_1 &= \frac{\pi^{2} \hbar^{2}}{2mL^{2}}\nonumber\\
\nonumber\\
E_2 &= \frac{4 \pi^{2} \hbar^{2}}{2mL^{2}} = 4E_{1}\nonumber\\
\nonumber\\
E_3 &= \frac{9 \pi^{2} \hbar^{2}}{2mL^{2}} = \boxed{9E_{1}}\nonumber
\end{align}
}


\Problem{45}{B}{%
First, solve for the $H \ket{n}$ from $n = 1$ to $n = 3$:

\begin{align}
H \ket{1} &= \frac{3}{2} \hbar \omega  \ket{1}\nonumber\\
\nonumber\\
H \ket{2} &= \frac{5}{2} \hbar \omega  \ket{2}\nonumber\\
\nonumber\\
H \ket{3} &= \frac{7}{2} \hbar \omega  \ket{3}\nonumber
\end{align}
\\
The inner product is therefore,

\begin{align}
\bra{\psi}H \ket{\psi} &= \frac{1}{14} \left(   \frac{3}{2} \hbar \omega\right)  + \frac{4}{14} \left(   \frac{5}{2} \hbar \omega\right) +  \frac{9}{14} \left(   \frac{7}{2} \hbar \omega \right) \nonumber\\
\nonumber\\
&= \left(   \frac{3}{28}  + \frac{20}{28}  + \frac{63}{28} \right) \hbar \omega = \frac{86}{28} \hbar \omega = \boxed{\frac{43}{14} \hbar \omega}   \nonumber
\end{align}
}



\Problem{46}{E}{%
The two equations that you need for this problem are the de Broglie formula and the kinetic energy equation:

\begin{gather}
\lambda = \frac{h}{p} \hspace{.1in} \\
T = \frac{1}{2}mv^{2} = \frac{p^{2}}{2m}
\end{gather}
\\
Noting that $h$ is a constant, we can rewrite the de Broglie formula:

\begin{gather}
\lambda_{0} p_{0} = h = \lambda_{f} p_{f} \hspace{.1in} \rightarrow \hspace{.1in} \lambda_{f} = \frac{\lambda_{0} p_{0}}{p_{f}} \nonumber
\end{gather}
\\
Then we can rewrite the kinetic energy equations:

\begin{gather}
T_{0} = E = \frac{p_{0}^{2}}{2m} \hspace{.1in} \rightarrow \hspace{.1in} p_{0} = \sqrt{2mE}\nonumber\\
T_{f} = E-V = \frac{p_{f}^{2}}{2m} \hspace{.1in} \rightarrow \hspace{.1in} p_{f} = \sqrt{2m(E-V)}\nonumber
\end{gather}
\\
Therefore,

\begin{gather}
\lambda_{1} = \frac{\lambda_{0} \sqrt{2mE}}{\sqrt{2m(E-V)}} = \lambda_{0}\sqrt{\frac{E}{E-V}} = \lambda_{0}\sqrt{\frac{1}{1-V/E}} = \boxed{\lambda_{0} \left(  1- V/E \right)^{-1/2}}\nonumber
\end{gather}
}



\Problem{47}{B}{%
From the second law of thermodynamics we know that

\begin{gather}
\Delta S = \int{\frac{dQ}{T}} = \frac{1}{T} \int{dQ}
\end{gather}
\\
because this container is sealed and thermally insulated $\Delta U = 0$. Therefore, from the first law of thermodynamics,   $d Q=dW$.

\begin{gather}
\Delta S = \int{\frac{dQ}{T}} = \frac{1}{T} \int{dW} = \frac{1}{T} \int{P dV} = nR \int{\frac{ dV}{V}}_{1}^{2} = nR\ln{(2/1)} = \boxed{n R \ln{(2)}}\nonumber
\end{gather}
}


\Problem{48}{C}{%
The rms velocity equation can be found by equating the kinetic energy of a gas particle to the total kinetic energy of the gas:

\begin{gather}
\frac{1}{2}mv^{2} = \frac{3}{2} k T \hspace{.1in} \rightarrow \hspace{.1in} v_{rms} = \sqrt{\frac{3kT}{m}}
\end{gather}
\\
Therefore with constant temperature, the ratio of $v_{rms}(N_{2})$ to $v_{rms}(O_{2})$ is

\begin{gather}
\frac{v_{rms}(N_{2})}{v_{rms}(O_{2})} = \frac{\sqrt{m_{O_{2}}}}{\sqrt{m_{N_{2}}}} = \frac{\sqrt{32}}{\sqrt{28}} = \boxed{\sqrt{\frac{8}{7}}}\nonumber
\end{gather}
}


\Problem{49}{E}{%
The partition function, $Z$, of a non-degenerate system is

\begin{gather}
Z = \sum_{n}{e^{-E_{n}/kT}} = e^{-\epsilon/kT} + e^{-2 \epsilon/kT}
\end{gather}
\\
Taking into account the degeneracy, $d$, of the system: the degenerate partition function, $Z_{D}$, is

\begin{gather}
Z_{D} = \sum_{n}{d_{n} e^{-E_{n}/kT}} =2 \left( e^{-\epsilon/kT} + e^{-2 \epsilon/kT}  \right)
\end{gather}
}


\Problem{50}{B}{%
From Wien's Law we know that $\lambda_{} T_{}  = const$. The wavelength is related to frequency by the equation  $\lambda_{} = c/\nu_{}$. Therefore

\begin{gather}
T_{1}  \frac{c_{1}}{\nu_{1}} = const = T_{2}  \frac{c_{2}}{\nu_{2}}\nonumber
\end{gather}
\\
From the problem, we know that $T_{1} = T_{2}$, $\nu_{1} = 440\Units{Hz}$, and $c_{2}  = 0.97 \cdot c_{1}$. Plugging in:

\begin{gather}
\frac{c_{1}}{\nu_{1}} = \frac{0.97 c_{1}}{\nu_{2}} \hspace{.1in} \rightarrow \hspace{.1in} \nu_{2} = 0.97 \cdot \nu_{1} = 0.97 \cdot 440\Units{Hz} \approx \boxed{427\Units{Hz}}\nonumber
\end{gather}
}


\Problem{51}{B}{%
The intensity light through the first polarizer, $I_{1}$, is half the the intensity of the initial beam:

\begin{gather}
I_{1} = \frac{I_{0}}{2}
\end{gather}
\\
Now, we must use Malus' Law, $I_{f} = I_{i} \cos^{2}{\theta}$ where $\theta = 45^{\circ}$ for the second polarizer and then $\theta = 45^{\circ}$ again for the third and final polarizer.

\begin{gather}
I_{2} = I_{1} \cos^{2}{45^{\circ}} = \frac{I_{1}}{2} = \frac{I_{0}}{4}\nonumber\\
I_{3} = I_{2} \cos^{2}{45^{\circ}} = \frac{I_{2}}{2} = \boxed{\frac{I_{0}}{8}}\nonumber
\end{gather}
}

\Problem{52}{B}{%
The volume of the primitive unit cell is the the volume of conventional unit cell divided by the number of lattice points in the Bravais lattice. A body-centered cubic lattice (BCC) has two lattice points (only count the sections inside the lattice) so that the volume is $\boxed{a^{3}/2}$\\
\\
In case you were wondering: A Bravais lattice is one that is isotropic at every point. A face centered cubic lattice (FCC) has four lattice points and a simple cubic lattice has one lattice point.
}

\Problem{53}{B}{%
Recall the resistivity equation from elementary electrodynamics:

\begin{gather}
R = R_{0}  \left( 1 + \alpha \Delta T  \right) \hspace{.1in} \rightarrow \hspace{.1in} \rho = \rho_{0}  \left( 1 + \alpha \Delta T  \right)
\end{gather}
\\
semiconductors have a negative coefficient of resistivity, $\alpha$, which means that the resistivity, $\rho$, should should decrease with increasing temperature. (B) choice
}

\Problem{54}{C}{%
The total impulse, $j$, delivered to a particle can be found using the equation:

\begin{gather}
j = \int{F \cdot dt}
\end{gather}
\\
Which is just the area under the curve of a Force vs time plot. The area is easy to calculate since the function forms a triangle:

\begin{gather}
j = \frac{1}{2}(2\Units{N} \cdot 2\Units{s}) = \boxed{2\Units{kg m/s}}\nonumber
\end{gather}
}

\Problem{56}{D}{%
This is a force balancing problem in which we imagine the ballon is stationary, i.e. the upward force is equal to the downward force.

\begin{align}
V g \rho_{air} &= mg + V g \rho_{He} \nonumber\\
\nonumber\\
V&= \frac{ m }{ \rho_{air} - \rho_{He} }  =  \frac{ 300\Units{kg}  }{ 1.29\Units{kg m$^{-3}$} - 0.18\Units{kg m$^{-3}$} } =
\boxed{270\Units{m$^{3}$} }\nonumber
\end{align}
}

\Problem{57}{A}{%
The momentum of the stream is

\begin{gather}
p = v dm= v \rho \cdot dV = v \rho \cdot A (dx) = v \rho \cdot A (v dt )\nonumber
\end{gather}
\\
We know from Newton's second law that $F =\dot{p}$ so that

\begin{gather}
F = \frac{d p}{d t} = \rho v^{2} A \frac{dt}{dt} =  \rho v^{2} A \nonumber
\end{gather}
}

\Problem{58}{B}{%
The Lorentz force equation is

\begin{gather}
\vec{F} = q (\vec{E} + \vec{v} \times \vec{B})
\end{gather}
\\
we can find the velocity, $v$ , of the proton by equating the particle's translational kinetic energy to its kinetic energy in a potential, $V \hat{z}$.

\begin{gather}
\frac{1}{2} m \vec{v}^{2} = q V \hat{z} \hspace{.1in} \rightarrow \hspace{.1in} \vec{v} = \sqrt{\frac{2qV}{m}}\hat{z}
\end{gather}
\\
The electric and magnetic fields are directed in the $+\hat{x}$ and $+\hat{y}$ direction, respectively, so that

\begin{align}
\vec{F} &= q (E\hat{x} + v\hat{z} \times B \hat{y}) \nonumber\\
&= q (E\hat{x} - vB\hat{x})
\end{align}
\\
When the particle's trajectory is not affected, $F = 0$ so that $E =vB$. However, if we are to increase the potential to $V' = 2V$ then

\begin{gather}
\vec{v'} = \sqrt{\frac{2qV'}{m}}\hat{z} = \sqrt{\frac{4qV}{m}}\hat{z} > \vec{v}
\end{gather}
\\
With this potential $E < vB$ and the resultant force is therefore the $\boxed{-\hat{x}}$ direction.
}

\Problem{59}{B}{%
The equation for simple harmonic motion of an oscillator is

\begin{gather}
m \frac{d^{2}x}{dt^{2}} + kx = 0
\end{gather}
\\
The analogous equation for the given circuit must have $\boxed{L = m\text{, }Q = x\text{, and }C = \frac{1}{k}}$
}


\Problem{60}{D}{%
The electric flux is

\begin{gather}
\Phi_{E} = \int{E \cdot dA} = \frac{Q}{\epsilon_{0}} = \frac{\sigma A}{\epsilon_{0}} = \frac{\sigma \pi c^{2}}{\epsilon_{0}}
\end{gather}
\\
where $c$ is the radius of the sheet's affected area and can be calculated via the pythagorean theorem:

\begin{gather}
R^{2} = c^{2} + x^{2} \hspace{.1in} \rightarrow \hspace{.1in} c^{2} = R^{2} - x^{2}\nonumber\\
\nonumber\\
\therefore \hspace{.1in} \boxed{\Phi_{E} = \frac{\sigma \pi \left(   R^{2} - x^{2}  \right)}{\epsilon_{0}}}\nonumber
\end{gather}
}





\end{document}

